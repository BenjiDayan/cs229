%
\documentclass{article}

\usepackage{graphicx}

\newcommand{\di}{{d}}
\newcommand{\nexp}{{n}}
\newcommand{\nf}{{p}}
\newcommand{\vcd}{{\textbf{D}}}

\usepackage{nccmath}
\usepackage{mathtools}
\usepackage{graphicx,caption}
\usepackage{enumitem}
\usepackage{epstopdf,subcaption}
\usepackage{psfrag}
\usepackage{amsmath,amssymb,epsf}
\usepackage{verbatim}
\usepackage[hyphens]{url}
\usepackage{color}
\usepackage{bbm}
\usepackage{listings}
\usepackage{setspace}
\usepackage{float}
\usepackage{natbib}
\definecolor{Code}{rgb}{0,0,0}
\definecolor{Decorators}{rgb}{0.5,0.5,0.5}
\definecolor{Numbers}{rgb}{0.5,0,0}
\definecolor{MatchingBrackets}{rgb}{0.25,0.5,0.5}
\definecolor{Keywords}{rgb}{0,0,1}
\definecolor{self}{rgb}{0,0,0}
\definecolor{Strings}{rgb}{0,0.63,0}
\definecolor{Comments}{rgb}{0,0.63,1}
\definecolor{Backquotes}{rgb}{0,0,0}
\definecolor{Classname}{rgb}{0,0,0}
\definecolor{FunctionName}{rgb}{0,0,0}
\definecolor{Operators}{rgb}{0,0,0}
\definecolor{Background}{rgb}{0.98,0.98,0.98}
\lstdefinelanguage{Python}{
numbers=left,
numberstyle=\footnotesize,
numbersep=1em,
xleftmargin=1em,
framextopmargin=2em,
framexbottommargin=2em,
showspaces=false,
showtabs=false,
showstringspaces=false,
frame=l,
tabsize=4,
% Basic
basicstyle=\ttfamily\footnotesize\setstretch{1},
backgroundcolor=\color{Background},
% Comments
commentstyle=\color{Comments}\slshape,
% Strings
stringstyle=\color{Strings},
morecomment=[s][\color{Strings}]{"""}{"""},
morecomment=[s][\color{Strings}]{'''}{'''},
% keywords
morekeywords={import,from,class,def,for,while,if,is,in,elif,else,not,and,or
,print,break,continue,return,True,False,None,access,as,,del,except,exec
,finally,global,import,lambda,pass,print,raise,try,assert},
keywordstyle={\color{Keywords}\bfseries},
% additional keywords
morekeywords={[2]@invariant},
keywordstyle={[2]\color{Decorators}\slshape},
emph={self},
emphstyle={\color{self}\slshape},
%
}


\pagestyle{empty} \addtolength{\textwidth}{1.0in}
\addtolength{\textheight}{0.5in}
\addtolength{\oddsidemargin}{-0.5in}
\addtolength{\evensidemargin}{-0.5in}
\newcommand{\ruleskip}{\bigskip\hrule\bigskip}
\newcommand{\nodify}[1]{{\sc #1}}
\newcommand{\points}[1]{{\textbf{[#1 points]}}}
\newcommand{\subquestionpoints}[1]{{[#1 points]}}
\newenvironment{answer}{{\bf Answer:} \sf \begingroup\color{red}}{\endgroup}%

\newcommand{\bitem}{\begin{list}{$\bullet$}%
{\setlength{\itemsep}{0pt}\setlength{\topsep}{0pt}%
\setlength{\rightmargin}{0pt}}}
\newcommand{\eitem}{\end{list}}

\setlength{\parindent}{0pt} \setlength{\parskip}{0.5ex}
\setlength{\unitlength}{1cm}

\renewcommand{\Re}{{\mathbb R}}
\newcommand{\R}{\mathbb{R}}
\newcommand{\what}[1]{\widehat{#1}}

\renewcommand{\comment}[1]{}
\newcommand{\mc}[1]{\mathcal{#1}}
\newcommand{\half}{\frac{1}{2}}

\def\KL{D_{KL}}
\def\xsi{x^{(i)}}
\def\ysi{y^{(i)}}
\def\zsi{z^{(i)}}
\def\E{\mathbb{E}}
\def\calN{\mathcal{N}}
\def\calD{\mathcal{D}}

\usepackage{tikz}
\usepackage{bbding}
\usepackage{pifont}
\usepackage{wasysym}
\usepackage{amssymb}
\usepackage{booktabs}
\usepackage{verbatim}


%\begin{document}
\begin{answer}
We see that for dataset A we quickly achieve convergence after 30374 iterations, whereas for dataset B it takes many more iterations to get $\theta$ to converge. For both the size of the gradient vector decreases over time, just much more slowly for dataset B.

\begin{verbatim*}
==== Training model on data set A ====
Finished 10000 iterations
Theta: [-20.81394174  21.45250215  19.85155266]
Size of theta: 1287.5141623056304
Size of grad: 5.222218480921017e-11
Finished 20000 iterations
Theta: [-20.81437785  21.45295156  19.85198173]
Size of theta: 1287.5686341528296
Size of grad: 2.8439949007650234e-19
Finished 30000 iterations
Theta: [-20.81437788  21.45295159  19.85198176]
Size of theta: 1287.568638172836
Size of grad: 1.5326309220390799e-27
Converged in 30374 iterations
\end{verbatim*}

Compared with

\begin{verbatim*}
==== Training model on data set B ====
Finished 10000 iterations
Theta: [-52.74109217  52.92982273  52.69691453]
Size of theta: 8360.153738379526
Size of grad: 0.001129658631566177
Finished 20000 iterations
Theta: [-68.10040977  68.26496086  68.09888223]
Size of theta: 13935.228454051608
Size of grad: 0.00047228214977839664
Finished 30000 iterations
Theta: [-79.01759142  79.17745526  79.03755803]
Size of theta: 18759.78475534314
...
\end{verbatim*}

It appears that with dataset B the theta vector is unhelpfully just growing in size, not really changing its direction at all.


\end{answer}
%\end{document}