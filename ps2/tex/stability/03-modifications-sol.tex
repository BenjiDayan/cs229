%
\documentclass{article}

\usepackage{graphicx}

\newcommand{\di}{{d}}
\newcommand{\nexp}{{n}}
\newcommand{\nf}{{p}}
\newcommand{\vcd}{{\textbf{D}}}

\usepackage{nccmath}
\usepackage{mathtools}
\usepackage{graphicx,caption}
\usepackage{enumitem}
\usepackage{epstopdf,subcaption}
\usepackage{psfrag}
\usepackage{amsmath,amssymb,epsf}
\usepackage{verbatim}
\usepackage[hyphens]{url}
\usepackage{color}
\usepackage{bbm}
\usepackage{listings}
\usepackage{setspace}
\usepackage{float}
\usepackage{natbib}
\definecolor{Code}{rgb}{0,0,0}
\definecolor{Decorators}{rgb}{0.5,0.5,0.5}
\definecolor{Numbers}{rgb}{0.5,0,0}
\definecolor{MatchingBrackets}{rgb}{0.25,0.5,0.5}
\definecolor{Keywords}{rgb}{0,0,1}
\definecolor{self}{rgb}{0,0,0}
\definecolor{Strings}{rgb}{0,0.63,0}
\definecolor{Comments}{rgb}{0,0.63,1}
\definecolor{Backquotes}{rgb}{0,0,0}
\definecolor{Classname}{rgb}{0,0,0}
\definecolor{FunctionName}{rgb}{0,0,0}
\definecolor{Operators}{rgb}{0,0,0}
\definecolor{Background}{rgb}{0.98,0.98,0.98}
\lstdefinelanguage{Python}{
numbers=left,
numberstyle=\footnotesize,
numbersep=1em,
xleftmargin=1em,
framextopmargin=2em,
framexbottommargin=2em,
showspaces=false,
showtabs=false,
showstringspaces=false,
frame=l,
tabsize=4,
% Basic
basicstyle=\ttfamily\footnotesize\setstretch{1},
backgroundcolor=\color{Background},
% Comments
commentstyle=\color{Comments}\slshape,
% Strings
stringstyle=\color{Strings},
morecomment=[s][\color{Strings}]{"""}{"""},
morecomment=[s][\color{Strings}]{'''}{'''},
% keywords
morekeywords={import,from,class,def,for,while,if,is,in,elif,else,not,and,or
,print,break,continue,return,True,False,None,access,as,,del,except,exec
,finally,global,import,lambda,pass,print,raise,try,assert},
keywordstyle={\color{Keywords}\bfseries},
% additional keywords
morekeywords={[2]@invariant},
keywordstyle={[2]\color{Decorators}\slshape},
emph={self},
emphstyle={\color{self}\slshape},
%
}


\pagestyle{empty} \addtolength{\textwidth}{1.0in}
\addtolength{\textheight}{0.5in}
\addtolength{\oddsidemargin}{-0.5in}
\addtolength{\evensidemargin}{-0.5in}
\newcommand{\ruleskip}{\bigskip\hrule\bigskip}
\newcommand{\nodify}[1]{{\sc #1}}
\newcommand{\points}[1]{{\textbf{[#1 points]}}}
\newcommand{\subquestionpoints}[1]{{[#1 points]}}
\newenvironment{answer}{{\bf Answer:} \sf \begingroup\color{red}}{\endgroup}%

\newcommand{\bitem}{\begin{list}{$\bullet$}%
{\setlength{\itemsep}{0pt}\setlength{\topsep}{0pt}%
\setlength{\rightmargin}{0pt}}}
\newcommand{\eitem}{\end{list}}

\setlength{\parindent}{0pt} \setlength{\parskip}{0.5ex}
\setlength{\unitlength}{1cm}

\renewcommand{\Re}{{\mathbb R}}
\newcommand{\R}{\mathbb{R}}
\newcommand{\what}[1]{\widehat{#1}}

\renewcommand{\comment}[1]{}
\newcommand{\mc}[1]{\mathcal{#1}}
\newcommand{\half}{\frac{1}{2}}

\def\KL{D_{KL}}
\def\xsi{x^{(i)}}
\def\ysi{y^{(i)}}
\def\zsi{z^{(i)}}
\def\E{\mathbb{E}}
\def\calN{\mathcal{N}}
\def\calD{\mathcal{D}}

\usepackage{tikz}
\usepackage{bbding}
\usepackage{pifont}
\usepackage{wasysym}
\usepackage{amssymb}
\usepackage{booktabs}
\usepackage{verbatim}


%\begin{document}
\begin{answer}
\begin{enumerate}
\item a higher constant learning rate would normally make convergence happen faster, but here there is no real convergence point. In fact the doubling time $n_k \propto \theta/(\alpha \text{grad})$ would decrease as expected, but the theta diff to determine stopping point is also increased, i.e. waiting for $\alpha \text{grad}$ small enough in size. I.e. overall the effect cancels out, given that $y \approx \propto 1/x$

\item Decreasing the learning rate over time would help speed stopping of the algorithm, e.g. $\alpha = 1/t^2$. This will definitely squeeze the size of the steps to quickly be small enough.

\item Linear scaling of the input features is unhelpful, it's akin to starting further or nearer along in the $\theta$ doubling process - doesn't remove the need for the increasingly lengthy doubling times

\item A regularization term of $||\theta||_2^2$ would help, as it would force an optimum to exist within an acceptable bounds of size of parameter, to which we could converge to much more naturally.

\item Adding zero-mean gaussian noise to he training data or labels would be useful, as either now the points aren't all split in half, or doubling $\theta$ endlessly is no longer a good strategy (some maximum $\theta$ over which we're counter productive).
\end{enumerate}
\end{answer}
%\end{document}
