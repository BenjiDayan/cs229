%\input{../macros.tex}
%\begin{document}
\begin{answer}
We get our plot of falling times for seeds 0,1,2,3
\begin{figure}[H]
	\includegraphics[width=15cm,height=9cm,keepaspectratio]{../src/cartpole/control_seed0.pdf}
	\caption{np.random.seed(0)}
\end{figure}
\begin{figure}[H]
	\includegraphics[width=15cm,height=9cm,keepaspectratio]{../src/cartpole/control_seed1.pdf}
	\caption{np.random.seed(1)}
\end{figure}
\begin{figure}[H]
	\includegraphics[width=15cm,height=9cm,keepaspectratio]{../src/cartpole/control_seed2.pdf}
	\caption{np.random.seed(2)}
\end{figure}
\begin{figure}[H]
	\includegraphics[width=15cm,height=9cm,keepaspectratio]{../src/cartpole/control_seed3.pdf}
	\caption{np.random.seed(3)}
\end{figure}


We have pretty different convergence behaviours for different random seeds. Max num trials survived is an order of magnitude different between e.g. seeds 1 and 2. Seed 1 had the best performance, although it took a while to converge fully (while maintaining that high performance) - I observed its value function difference spiking at multiple occasions which prevented the no learning threshold of 20 from triggering a few times.
\end{answer}
%\end{document}
